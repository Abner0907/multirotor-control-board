\documentclass[11pt, a4paper]{article}

\usepackage[czech]{babel}
\usepackage[IL2]{fontenc} % pro iso8859-2
\usepackage[utf8]{inputenc}   % pro unicode UTF-8

\newcommand{\Author}{Tomáš Báča}
\newcommand{\Title}{Řídicí deska multikopter}
\newcommand{\Acronym}{Acronym}
\newcommand{\WorkPackage}{WorkPackage}
\newcommand{\DocName}{Technický manuál}
\newcommand{\Subject}{\WorkPackage - \DocName}
\newcommand{\Keywords}{mobile robotics}
\newcommand{\Date}{11/07/2011}
\newcommand{\DOCVersion}{0.1}

\pdfoutput=1
\documentclass[a4paper,12pt,titlepage, twoside]{article}


\usepackage[english]{babel}
\usepackage[utf8]{inputenc}
\usepackage{amssymb,amsmath}
\usepackage{algorithm,algpseudocode}
\usepackage[title,titletoc]{appendix}
\usepackage{latexsym}
\usepackage{a4wide}
\usepackage{color} 
\usepackage{indentfirst}
\usepackage{graphicx}       %%% graphics for dvips
\usepackage{fancyhdr}
\usepackage{longtable}
\usepackage{pifont}
\usepackage{makeidx}
\usepackage{lastpage}
\usepackage{multirow}
\usepackage{dcolumn} 
\usepackage{epstopdf}
\usepackage{url}
\usepackage{listings}
\usepackage{caption}
\usepackage{subcaption}
\usepackage{relsize}
\usepackage{pdfpages}
\usepackage{rotating}
\usepackage{natbib}
\usepackage{cite}
\usepackage{booktabs}
\usepackage[table,xcdraw]{xcolor}

\newcommand{\Author}{Jiří Fiedler}
\newcommand{\Title}{MAV communication protocol}
\newcommand{\Acronym}{Acronym}
\newcommand{\WorkPackage}{WorkPackage}
\newcommand{\DocName}{MAV communication protocol}
\newcommand{\Subject}{\WorkPackage - \DocName}
\newcommand{\Keywords}{mobile robotics}
\newcommand{\Date}{11/05/2015}
\newcommand{\DOCVersion}{1.0}

% European layout (no extra space after `.')
\frenchspacing

% nastavení výstupu
\def\nothtml{}  					%%% \nothtml is defined if not processed with latex2html
\usepackage[                		%%% hyper-references for ps2pdf
bookmarks=true,%                   	%%% generate bookmarks ...
breaklinks=true,%                  	%%% breaks lines, but links are very small
hypertexnames=false,%              	%%% needed for correct links to figures
colorlinks=false,%
urlcolor=blue
]{hyperref}           				%%% blue instead of cyan URLS
\hypersetup{
pdfcreator  = {LaTeX with hyperref package},
pdfproducer = {dvips + ps2pdf},
colorlinks=false,
pdfborder={0 0 0},
}

\hypersetup{  
pdfauthor={\Author},
pdftitle={\Title - \Acronym},
pdfsubject={\Subject},
pdfkeywords={\Keywords}}

% úpravy vzhledu stránek
\setlength{\headheight}{18pt}			%%% drobně odsadí hlavičku
\renewcommand{\footrulewidth}{0.4pt}  	%%% horizontal line in footer
\fancyhead[R]{} 						%%% umaže pravou stranu hlavičky

\newcommand{\jed}[1]{\ensuremath{~\mathrm{#1}}} %příkaz pro sazbu fyzikálních jednotek
\newcommand{\dd}[1]{\ensuremath{\mathrm{d}#1}} %příkaz pro sazbu diferenciálu
\newcommand{\EE}[1]{\ensuremath{ \cdot 10^{#1}}} %příkaz pro sazbu *10^x

\newcommand{\pd}[2]{\ensuremath{\frac{\partial #1}{\partial #2}}} %parc. derivačka

\begin{document}

\begin{titlepage}

\begin{center}

\textsc{\LARGE České vysoké učení technické v Praze }\\[1.0cm]
\textsc{\LARGE Fakulta elektrotechnická }\\[1.0cm]
\textsc{\large Skupina inteligentní a mobilní robotiky }\\[0.5cm]

% Title
%\HRule \\[5.0cm]
{ \huge \bfseries Technický manuál\\[0.5cm] řídicí deska multicopter }\\[1.0cm]
%\HRule \\[1.5cm]

% Author and supervisor
\large
\textsc{\Author}

\vfill

\end{center}

\end{titlepage}

\newpage

\tableofcontents

\newpage

\setlength{\parskip}{0.35cm}
\pagenumbering{arabic}
\lhead{\emph{\leftmark}}
\rhead{}
\cfoot{}
\rfoot{\thepage$/$\pageref{LastPage}}

\section{Hardware desky}

Řídicí deska plní funkci komunikačního uzlu mezi moduly a malé výpočetní jednotky pro stabilizaci a řízení letounu. Je osazena mikrokontrolerem ATmega164p s 1kb RAM a 16kb ROM (pro program). Procesor kontroleru trpí absencí FPU (floating point unit). Ke komunikaci jsou zde 2 porty UART. Programování mikrokontroleru probíhá přes SPI.

Deska má 9 vstupů pro PWM. Použity jsou pro příjem signálů z RC přijímače. Dále jsou zde dva volitelné výstupy, z nichž jeden je použit pro PPM výstup do stabilizační desky letounu a druhý je volitelný.

Pro signalizaci jsou k dispozici 2 LED, žlutá a červená, třetí se dá případně připojit do volitelného výstupu.

Deska je osazena dvěma volitelnými tlačítky a jedním tlačítkem \textbf{reset}.

\section{Software}

\subsection{Struktura software}

Software pro kontroler je napsán v programovacím jazyce C. Hlavním souborem je \textbf{main.c}. Zde je hlavní smyčka programu a předpis funkcí pro přerušení přerušeními procesoru.

Většinu času tráví procesor v nekonečné smyčce ve funkci \textbf{main()}. Zde čeká na asynchronní obsluhu komunikace, nebo periodické volání řídicích funkcí apod. Je velmi důležité, aby co možná všechny výpočty, zpracování a volání funkcí probíhaly voláním z funkce \textbf{main()}. Jinak bude procesor zablokován (přerušení nebude vyvoláno, pokud procesor vykonává funkci jiného přerušení) a nebude zpracovávat komunikaci, což může vést k neovladatelnosti letounu.

Kód je strukturován pomocí podmínek preprocesoru. Pokud např. nebudete potřebovat kamerový modul s počítačem gumstix, lze v souboru \textbf{config.h} definovat \texttt{GUMSTIX\_DATA\_REVEICE} na hodnotu \texttt{DISABLED}. Všechen kód a proměnné týkající se příjmu dat a volání funkcí, s tímto modulem spojených, bude vyřazeno z kompilace.

\subsubsection{main.c}

Kód \textbf{main.c} obsahuje definici globálních proměnných. Všechny proměnné, které si mají uchovat hodnotu, musí být deklarovány zde. Všechny proměnné doporučuji definovat jako \textbf{volatile}, předejdete nečekaným problémům s jejich měnícím se obsahem.

Je zde definována funkce \textbf{main()} v níž probíhá konfigurace procesoru po spuštění a následně zanoření do nekonečné smyčky.

Dále se zde nachází obsluha přerušení, viz. Tabulka~\ref{tab:interrupts}.

V nekonečné smyčce jsou kontrolovány vlajky pro asynchronní volání funkcí. Např., poté, co přijde poslední znak zprávy z px4flow, nastaví se vlajka \textbf{px4flowDataFlag} na hodnotu 1. Ta je poté v nekonečné smyčce odchycena a jsou vykonány potřebné úkony spojené s příjmem dat - filtrace, uložení aktuálních hodnot do stavových proměnných, apod.

\begin{table}
\begin{center}
\begin{tabular}{| c | p{8cm} |}
\hline
ISR(USART0\_RX\_vect) & příjem z UART0 \\
\hline
ISR(USART1\_RX\_vect) & příjem z UART1 \\
\hline
ISR(TIMER1\_COMPA\_vect) & Komparační přerušení A k 16bit čítači. Používá se pro generování vstupního PPM signálu (start pulzu).
\newline \textbf{Neupravovat, pokud nevím, co dělám!!}\\
\hline
ISR(TIMER1\_COMPB\_vect) & Komparační přerušení B k 16bit čítači. Používá se pro generování vstupního PPM signálu (konec pulzu).
\newline \textbf{Neupravovat, pokud nevím, co dělám!!}\\
\hline
ISR(PCINT0\_vect) & Zpracování změny na vstupních PWM pinech 1..4
\newline \textbf{Neupravovat, pokud nevím, co dělám!!}\\
\hline
ISR(PCINT1\_vect) & Zpracování změny na vstupních PWM pinech 5..9
\newline \textbf{Neupravovat, pokud nevím, co dělám!!}\\
\hline
ISR(TIMER0\_OVF\_vect) & Zpracování přetečení 8bit čitače (hrubé měření času). \\
\hline
\end{tabular}
\caption{Použitá přerušení}
\label{tab:interrupts}
\end{center}
\end{table}

\subsubsection{controllers.c}

\textbf{controllers.c} obsahuje funkce pro řízení letounu.

\subsubsection{system.c}

\textbf{system.c} obsahuje funkce pro obsluhu kontroleru a letounu. Viz. Tabulka~\ref{tab:system.c}.

\begin{table}
\begin{center}
\begin{tabular}{| c | p{8cm} |}
\hline
initializeMCU() & Volá se jednou, po zapnutí. Provede nastavení mikrokontroleru.
\newline \textbf{Neupravovat, pokud nevím, co dělám!!}\\
\hline
enableController() & Zapne automatickou stabilizaci letounu.\\
\hline
disableController() & Vypne automatickou stabilizaci letounu.\\
\hline
armVehicle() & Provede automatické "armování" letounu. NEPOUŽÍVAT!.\\
\hline
disarmVehicle() & Provede automatické "disarmování" letounu.\\
\hline
button1check() & Kontrola stisku tlačítka 1.\\
\hline
button2check() & Kontrola stisku tlačítka 2.\\
\hline
\end{tabular}
\caption{Funkce v system.c}
\label{tab:system.c}
\end{center}
\end{table}

\subsubsection{communication.c}

\textbf{communication.c} obsahuje funkce pro obsluhu a zpracování komunikace s externími moduly. Některé funkce zpracovávají jednotlivé příchozí bajty a jsou tedy určeny pro volání zevnitř přerušení. Jiné zpracovávají celou komunikační zprávu a musejí být volání asynchronně, zevnitř smyčky, po přijetí posledního bajtu.

\begin{table}
\begin{center}
\begin{tabular}{| c | p{8cm} |}
\hline
USART0\_init() & inicializace UART0\\
\hline
USART1\_init() & inicializace UART1\\
\hline
Uart0\_write\_char() & zapíše bajt na UART0\\
\hline
Uart0\_write\_string() & zapíše string na UART1\\
\hline
atomParseChar() & zpracuje příchozí znak z "Atomového" počítače (použití pro surfnav)\\
\hline
gumstixParseChar() & zpracuje příchozí znak z Gumstixu\\
\hline
flightCtrlParseChar() & zpracuje příchozí znak z FlightCTRL stabilizační desky\\
\hline
Decode64() & dekóduje zprávu z FlightCTRL\\
\hline
parseFlightCtrlMessage() & zpracuje zprávu z FlightCTRL stabilizační desky\\
\hline
mergeSignalsToOutput() & provádí míchání signálů regulátorů a RC vysílače. \newline \textbf{Neupravovat, pokud nevím, co dělám!!}\\
\hline
px4flowParseChar() & zpracuje příchozí znak ze senzoru px4flow\\
\hline
my\_mavlink\_parse\_char() & upravená funkce z knihovny MavLink. zpracuje znak ze senzoru px4flow.\\
\hline
capturePWMInput() & měří délku PWM pulzů z RC soupravy.\newline \textbf{Neupravovat, pokud nevím, co dělám!!}\\
\hline
\end{tabular}
\caption{Funkce v communication.c}
\label{tab:communication.c}
\end{center}
\end{table}

\subsubsection{config.h}

\textbf{config.h} obsahuje direktivy preprocesoru pro konfiguraci celého firmwaru. Lze zde zapínat jednotlivé moduly (px4flow, gumstix, atomový PC, FligthCTRL) a nastavovat chování firmwaru. Dále se zde konfigurují seriové linky (jejich BAUD rate) a jejich přiřazení k modulům.

Důležité je nastavení konstant pro PWM a PPM. Jsou zde hodnoty délek pulzů (min, střední, max), délka PPM rámce a délka dělícího pulzu v PPM.

Dále je zde namapování kanálů z RC na jednotlivé PWM vstupy.

\begin{table}
\begin{center}
\begin{tabular}{| c | p{8cm} |}
\hline
FRAME\_ORIENTATION & Nastavení orientace letounu (PLUS\_COPTER, nebo X\_COPTER). Má vliv na úhly vyčítané z FlightCTRL, ty jsou vždy relativně k desce FlightCTRL. \\
\hline
GUMSTIX\_CAMERA\_POINTING & Kam míří kamera Gumstix počítače (FORWARD, nebo DOWNWARD), důležité pro rotaci souřadnic.\\
\hline
disableController() & Vypne automatickou stabilizaci letounu.\\
\hline
armVehicle() & Provede automatické "armování" letounu. NEPOUŽÍVAT!.\\
\hline
disarmVehicle() & Provede automatické "disarmování" letounu.\\
\hline
button1check() & Kontrola stisku tlačítka 1.\\
\hline
button2check() & Kontrola stisku tlačítka 2.\\
\hline
\end{tabular}
\caption{Obsah config.h}
\label{tab:config.h}
\end{center}
\end{table}

\subsection{Kompilace pro ATmega164p}

Ke kompilaci je třeba mít nainstalovaný kompilátor \textbf{avr-gcc}. Pro windows je obsažen v balíčku \textbf{WinAVR}, ke stažení na adrese \url{http://winavr.sourceforge.net/}. Pro Linux je potřeba balíčky \textbf{gcc-avr}, \textbf{binutils-avr}, \textbf{avr-libc}, \textbf{avrdude}, \textbf{gdb-avr}.

\texttt{sudo apt-get install gcc-avr binutils-avr gdb-avr avr-libc avrdude}

Pro samotnou kompilaci je přítomen soubor Makefile, který obsahuje předpis pro všechny výše popsané soubory. Výstupem kompilace je soubor \textbf{main.hex}, který je vstupním parametrem pro upload do mikrokontroleru. V Linuxu je kompilace prostá - zavoláním příkazu \texttt{make~all} ve složce se zdrojovými soubory. Ve Windows je postup stejný, pokud jste si nainstalovali výše zmíněný toolchain. Ten do windows doinstaloval program make, který umí interpretovat Makefile stejně, jako v Linuxu.

\subsection{Upload progamu do ATmega164p}

Po úspěšné kompilaci máte ve složce se zdrojovými soubory soubor \textbf{main.hex}. K jeho nahrání do kontroleru potřebujete programátor (např. USBasp) a program \textbf{avrdude}. Příkaz pro upload souboru je shodný pro Windows i Linux, pouze v Linuxu je nutné volat ho s právy superuživatele.

\texttt{avrdude -p m164P -c usbasp -U flash:w:main.hex}

\textbf{POZOR!} Před nahráváním si vždy zkontrolujte, zdali je do řídicí desky řádně přivedeno napájení. Pokud je deska bez napájení, může nahrávání firmwaru nenávratně poškodit kontroler.

\end{document}